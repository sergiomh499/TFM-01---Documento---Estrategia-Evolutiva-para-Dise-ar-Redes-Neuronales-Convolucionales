% !TEX root = my-thesis.tex

% **************************************************
% Extra packages
% **************************************************
\usepackage{multicol}
%\usepackage{epstopdf}
%\setlength{\belowcaptionskip}{-10pt}
\usepackage{booktabs}

% Para hacer una pagina horizontal. Uso: \begin{landscape} xxxx \end{lanscape}
\usepackage{lscape} 
% Para incluir paginas PDF. Uso:
% \includepdf[pages={1}]{tuarchivo.pdf}
\usepackage{pdfpages}
% Para introducir url's con formato. Uso: \url{http://www.google.es}
\usepackage{url}
% Amplia muchas funciones graficas de latex
\usepackage{graphicx}
% Paquete que añade el hipervinculo en referencias dentro del documento, indice, etc
% Se define sin bordes alrededor. Uso: \ref{tulabel}
\usepackage{hyperref}
%\usepackage[pdfborder={000}]{hyperref}
\usepackage{float}
\usepackage{placeins}
\usepackage{afterpage}
\usepackage{verbatim}
% Paquete para condicionales avanzados
\usepackage{xstring,xifthen}
% Paquete para realizar cálculos en el código
\usepackage{calc}
% Para rotar tablas o figuras o su contenido
\usepackage{rotating} 
% Para incluir comentarios en el texto. El parámetro 'disable' oculta todas las notas.
% USO: \todo{tutexto}
\usepackage[textsize=tiny,spanish,shadow,textwidth=2cm]{todonotes}

\usepackage{subcaption} % Para poder realizar subfiguras
%\usepackage{caption} % Para aumentar las opciones de diseño
% Nombres de figuras, tablas, etc, en negrita la numeración, todo con letra small
\captionsetup{labelfont={bf,small},textfont=small}
% Paquete para modificar los espacios arriba y abajo de una figura o tabla
\usepackage{setspace}
% Define el espacio tanto arriba como abajo de las figuras, tablas
\setlength{\intextsep}{5mm}
% Para ajustar tamaños de texto de toda una tabla o grafica
% Uso: {\scalefont{0.8} \begin{...} \end{...} }
\usepackage{scalefnt}

\usepackage{mathtools}
\DeclarePairedDelimiter\ceil{\lceil}{\rceil}
\DeclarePairedDelimiter\floor{\lfloor}{\rfloor}
\DeclareMathOperator*{\argmin}{arg\,min}
\providecommand{\abs}[1]{\lvert#1\rvert}

\usepackage{svg}

\usepackage{multirow}

\usepackage[12pt]{moresize}


% **************************************************
% Files' Character Encoding
% **************************************************
\PassOptionsToPackage{utf8}{inputenc}
\usepackage{inputenc}


% **************************************************
% Information and Commands for Reuse
% **************************************************
\newcommand{\thesisTitle}{Estrategia Evolutiva para Diseñar Redes Neuronales Convolucionales}
\newcommand{\thesisName}{Sergio Martínez Hamdoun}
\newcommand{\thesisSubject}{Trabajo Fin de Máster}
\newcommand{\thesisDate}{28 de Junio de 2021}
\newcommand{\thesisVersion}{Documento Final}

\newcommand{\thesisFirstReviewer}{Ángela María Ribeiro Seijas}
\newcommand{\thesisFirstReviewerUniversity}{\protect{Universidad Politécnica de Madrid}}
\newcommand{\thesisFirstReviewerDepartment}{Centro de Automática y Robótica}

\newcommand{\thesisSecondReviewer}{Jeremy Karouta}
\newcommand{\thesisSecondReviewerUniversity}{\protect{Universidad Politécnica de Madrid}}
\newcommand{\thesisSecondReviewerDepartment}{Centro de Automática y Robótica}

\newcommand{\thesisFirstSupervisor}{Jane Doe}
\newcommand{\thesisSecondSupervisor}{John Smith}

\newcommand{\thesisUniversity}{\protect{Universidad Politécnica de Madrid}}
\newcommand{\thesisUniversityDepartment}{Department of Clean Thesis Style}
\newcommand{\thesisUniversityInstitute}{Institute for Clean Thesis Dev}
\newcommand{\thesisUniversityGroup}{Clean Thesis Group (CTG)}
\newcommand{\thesisUniversityCity}{Madrid}
\newcommand{\thesisUniversityStreetAddress}{Street address}
\newcommand{\thesisUniversityPostalCode}{Postal Code}


% **************************************************
% Debug LaTeX Information
% **************************************************
%\listfiles

% **************************************************
% Load and Configure Packages
% **************************************************
\usepackage[spanish]{babel} % babel system, adjust the language of the content
\PassOptionsToPackage{% setup clean thesis style
    figuresep=endash,%
    sansserif=false,%
    hangfigurecaption=true,%
    hangsection=true,%
    hangsubsection=true,%
    colorize=full,%
    colortheme=bluegreen,% bluemagenta
    bibsys=bibtex,%
    bibfile=bib-refs,%
    bibstyle=numeric ,%
    wrapfooter=false,%
}{cleanthesis}
\usepackage{cleanthesis}

\hypersetup{% setup the hyperref-package options
    pdftitle={\thesisTitle},    %   - title (PDF meta)
    pdfsubject={\thesisSubject},%   - subject (PDF meta)
    pdfauthor={\thesisName},    %   - author (PDF meta)
    plainpages=false,           %   -
    colorlinks=false,           %   - colorize links?
    pdfborder={0 0 0},          %   -
    breaklinks=true,            %   - allow line break inside links
    bookmarksnumbered=true,     %
    bookmarksopen=true          %
}


