% !TEX root = ../my-thesis.tex
%
\chapter{CONCLUSIONES Y TRABAJO FUTURO}
\label{sec:conclusiones}

\section{Conclusiones}

Con la elaboración de este trabajo se perseguía la mejora del proceso de clasificación y localización de las conocidas como malas hierbas en cultivos, para su posterior tratamiento empleando técnicas basadas en la agricultura de precisión, que buscan un mayor rendimiento de los cultivos, disminuyendo los costes y siendo más sostenibles con el medioambiente.

Buscando este fin, se proponía un sistema alternativo a los métodos más empleados en la actualidad, los cuáles se basan en la implementación de CNNs con arquitecturas genéticas pre-entrenadas, las que a su vez presentaban grandes inconvenientes como son su gran tamaño y el gran coste de entrenamiento necesario para estas.

Este sistema alternativo se basa en la hipótesis de que, empleando técnicas de optimización, se podrían generar nuevas arquitecturas de CNNs de manera automática que resolvieran este problema de clasificación con un menor tiempo de entrenamiento, siendo de menor tamaño y teniendo las mismas capacidades que las redes actualmente empleadas. Para esto, se propone el uso de técnicas evolutivas, las cuáles generarán y optimizarán estas redes de forma eficiente, consiguiendo validar la hipótesis propuesta.

Se desarrolló, para ello, una implementación basada en el uso de Algoritmos Genéticos que generan diferentes arquitecturas que compiten por tener un mejor desempeño en la clasificación de un conjunto de imágenes de malas hierbas dado. Esta fue modificada y adaptada de tal manera que se encontrara el mayor rendimiento del algoritmo, tratando de obtener arquitecturas con un gran rendimiento.

Mediante este sistema, se han conseguido obtener una serie de individuos con una alta capacidad de clasificación del \textit{dataset} de entrada. Estas tienen un poder de clasificación superior al 90\% de acierto situándose en valores muy cercanos a los obtenidos con la popular arquitectura ResNet-50, empleando imágenes de entrada con la mitad de resolución y con arquitecturas que emplean 10 veces menos cantidad de capas, una novena parte de parámetros y cerca de la mitad de tiempo de entrenamiento.

Por tanto, concluyo basándome en los resultados obtenidos y tomando la hipótesis de partida como válida, ya que se han podido encontrar arquitecturas que se adaptan de manera más eficiente al problema de clasificación dado, consiguiendo un rendimiento similar a arquitecturas como ResNet-50.


\section{Líneas Futuras}

Finalmente, tras haber obtenido unos resultados realmente prometedores, se cree que aún queda mucho potencial por extraer del desarrollo realizado en este trabajo, por lo que a continuación, se propondrán una serie de desarrollos futuros que serían de gran interés.

\begin{itemize}
    \item Debido a la extensión del trabajo y la duración temporal del mismo, resulta imposible mantener al algoritmo ejecutándose durante el tiempo que este puede requerir. Es por esto, que sería interesante lanzar \textbf{ejecuciones de tiempo superior} a los 18 días actuales, para asegurar que el espacio de búsqueda ha sido ampliamente analizado y se ha llegado a un máximo dentro de este.
    
    \item Observando que el tiempo de ejecución es una limitación muy fuerte dentro de la ejecución del algoritmo, sería adecuado proponer soluciones al gran cuello de botella existente en este, que es la duración de los entrenamientos de las diferentes arquitecturas generadas. Para esto, se propone la ejecución en una infraestructura que permita de forma adecuada la \textbf{paralelización del entrenamiento} de estas, de manera que se entrenen varias redes de forma simultanea en diferentes GPUs.
    
    \item En este trabajo, se ha comprobado como existen individuos con gran desempeño y con un tamaño bastante reducido. Esto se ha conseguido marcando cómo único objetivo del Algoritmo Genético el maximizar el valor de \textit{accuracy} de las arquitecturas generadas. Es por ello, que sería adecuado la \textbf{implementación de técnicas evolutivas multi-objetivo} donde se persiga no solo maximizar el desempeño de la red, sino que además busque reducir alguna variable de interés como puede ser el tiempo por \textit{epoch}, el número de capas o el número de parámetros.
    
    \item Bien es cierto que en este trabajo se conseguido clasificar con éxito entre 2 tipos diferentes de malas hierbas de otras que no lo eran. Esto se realizó así debido a las limitaciones temporales que impone la ejecución del algoritmo completo. Es por ello, que de implementarse algunas de las técnicas anteriormente comentadas, sería adecuado probar con \textbf{conjuntos de datos que contengan más clases a clasificar}, de manera que se compruebe que esta técnica funciona de manera adecuada para conjuntos más amplios de imágenes.
    
    \item La actual codificación desarrollada es capaz de generar diferentes topologías de redes, entre las que se encuentran algunas de las más empleadas en la actualidad. Aún así, existen diferentes redes, que aún no pueden ser generadas con esta, como puede ser por ejemplo las arquitecturas con bloques tipo Inception. Es por ello, que se cree adecuado, extender la codificación actual, para \textbf{permitir la codificación de nuevas relaciones entre las diferentes capas}, lo cuál extraerá soluciones más complejas y posiblemente con mayor potencial.
    
\end{itemize}

