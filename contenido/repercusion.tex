% !TEX root = ../my-thesis.tex
%

\chapter{RESPONSABILIDAD AMBIENTAL, SOCIAL Y ECONÓMICA}
\label{sec:repercusion}

En la elaboración de este Trabajo Fin de Máster se desarrollan numerosas temáticas que atañen a aspectos sociales, medioambientales y económicos. Esto es así, debido a que se engloba dentro de la mejora de procesos en temas de agricultura, más específicamente, en técnicas de detección de malas hierbas dentro de la agricultura de precisión. Como se ha podido ver en los capítulos anteriores, esta busca dar solución a las preocupaciones actuales del hombre sobre el crecimiento de la población, la falta de alimento o la protección del medioambiente.

Este trabajo se apoya en el uso de las nuevas tecnologías para la implantación de nuevas técnicas dentro del campo, haciendo las cosechas más productivas, evitando el exceso de uso de agentes externos que usados de forma extensiva pueden empeoran la calidad del producto, disminuir el rendimiento del cultivo o contaminar espacios naturales que no eran objetivo de estas.

Más específicamente, desde el lado medioambiental, la implantación de este tipo de técnicas vienen acompañadas de mejoras claras en la forma que son tratados los actuales cultivos. Esto es así, debido a que de aplicarse de forma adecuada, se conocerán las necesidades prácticamente individualizadas de cada una de las plantas, lo que permite, por ejemplo, la detección de plagas o enfermedades de forma prematura. Además, conociendo de que manera y dónde se aplican ciertos químicos, evitan la posible contaminación de otros espacios naturales que no son objetivo de estas, como pueden ser acuíferos cercanos u otras cosechas.

Desde el punto de vista social, las soluciones que soportan en este trabajo, persiguen una mejora en las condiciones laborales de los trabajadores del campo, automatizando procesos que actualmente deben realizarse completamente de forma manual. Además, se persigue que la mejora en la producción y en la calidad de los alimentos, disminuya los costes de producción de forma drástica y consiguiendo el acceso a los alimentos de forma más sencilla en países donde esto no está asegurado.

Esto, influye drásticamente también en el lado económico, ofreciendo nuevas tecnologías que cada vez son más accesibles a numerosas áreas, donde estas herramientas pueden favorecer en gran medida la mejora de estas. Esto a su vez, hace que los costes de producción y la adquisición e implantación de este tipo de tecnologías sean cada vez más bajos.

Finalmente, este capítulo se encuentra claramente ligado con los Objetivos de Desarrollo Sostenible (ODS) impulsados por la Organización de las Naciones Unidas (ONU). Es por ello, que este trabajo se puede enmarcar dentro de los siguientes objetivos junto a las metas a perseguir:

\begin{itemize}
    \item \textbf{Objetivo 2. Hambre Cero.}
    \begin{itemize}
        \item Meta 2.1. Para 2030, poner fin al hambre y asegurar el acceso de todas las personas, en particular los pobres y las personas en situaciones vulnerables, incluidos los lactantes, a una alimentación sana, nutritiva y suficiente durante todo el año.
        \item Meta 2.3. Para 2030, duplicar la productividad agrícola y los ingresos de los productores de alimentos en pequeña escala, en particular las mujeres, los pueblos indígenas, los agricultores familiares, los pastores y los pescadores, entre otras cosas mediante un acceso seguro y equitativo a las tierras, a otros recursos de producción e insumos, conocimientos, servicios financieros, mercados y oportunidades para la generación de valor añadido y empleos no agrícolas.
        \item Meta 2.4. Para 2030, asegurar la sostenibilidad de los sistemas de producción de alimentos y aplicar prácticas agrícolas resilientes que aumenten la productividad y la producción, contribuyan al mantenimiento de los ecosistemas, fortalezcan la capacidad de adaptación al cambio climático, los fenómenos meteorológicos extremos, las sequías, las inundaciones y otros desastres, y mejoren progresivamente la calidad del suelo y la tierra.
    \end{itemize}
    \item \textbf{Objetivo 3. Salud y Bienestar.}
    \begin{itemize}
        \item Meta 3.9. Para 2030, reducir sustancialmente el número de muertes y enfermedades producidas por productos químicos peligrosos y la contaminación del aire, el agua y el suelo.
    \end{itemize}
    \item \textbf{Objetivo 6. Agua Limpia y Saneamiento.}
    \begin{itemize}
        \item Meta 6.3. De aquí a 2030, mejorar la calidad del agua reduciendo la contaminación, eliminando el vertimiento y minimizando la emisión de productos químicos y materiales peligrosos, reduciendo a la mitad el porcentaje de aguas residuales sin tratar y aumentando considerablemente el reciclado y la reutilización sin riesgos a nivel mundial.
        \item Meta 6.4. De aquí a 2030, aumentar considerablemente el uso eficiente de los recursos hídricos en todos los sectores y asegurar la sostenibilidad de la extracción y el abastecimiento de agua dulce para hacer frente a la escasez de agua y reducir considerablemente el número de personas que sufren falta de agua.
        \item 6.6. De aquí a 2020, proteger y restablecer los ecosistemas relacionados con el agua, incluidos los bosques, las montañas, los humedales, los ríos, los acuíferos y los lagos.
    \end{itemize}
    \item \textbf{Objetivo 8. Trabajo Decente y Crecimiento Económico.}
    \begin{itemize}
        \item Meta 8.2. Lograr niveles más elevados de productividad económica mediante la diversificación, la modernización tecnológica y la innovación, entre otras cosas centrándose en los sectores con gran valor añadido y un uso intensivo de la mano de obra.
        \item Meta 8.4. Mejorar progresivamente, de aquí a 2030, la producción y el consumo eficientes de los recursos mundiales y procurar desvincular el crecimiento económico de la degradación del medio ambiente, conforme al Marco Decenal de Programas sobre modalidades de Consumo y Producción Sostenibles, empezando por los países desarrollados
        \item Meta 8.8. Proteger los derechos laborales y promover un entorno de trabajo seguro y sin riesgos para todos los trabajadores, incluidos los trabajadores migrantes, en particular las mujeres migrantes y las personas con empleos precarios.
    \end{itemize}
    \item \textbf{Objetivo 12. Producción y Consumos Responsable}
    \begin{itemize}
        \item Meta 12.2. De aquí a 2030, lograr la gestión sostenible y el uso eficiente de los recursos naturales.
        \item Meta 12.4. De aquí a 2020, lograr la gestión ecológicamente racional de los productos químicos y de todos los desechos a lo largo de su ciclo de vida, de conformidad con los marcos internacionales convenidos, y reducir significativamente su liberación a la atmósfera, el agua y el suelo a fin de minimizar sus efectos adversos en la salud humana y el medio ambiente.
    \end{itemize}
\end{itemize}
