% !TEX root = ../my-thesis.tex
%
% \pdfbookmark[0]{Resumen}{Resumen}
\chapter*{Resumen}
\addcontentsline{toc}{chapter}{RESUMEN}
\label{sec:abstract}
\vspace*{-10mm}
% ------------------------------------------------------------------------------

La agricultura extensiva en la actualidad, es una actividad imprescindible, encargada de proporcionar alimentación a la población mundial. Pero esta, presenta graves problemas de carácter medioambiental, de cara a la conservación del planeta, como son la contaminación de las aguas cercanas por empleo indiscriminado de agentes químicos para el tratamiento plagas y de las denominadas malas hierbas. Además de este, presenta otros problemas, como grandes costes asociados a estas técnicas y rendimientos mejorables de las cosechas.

Para ello, la agricultura de precisión propone técnicas por las cuales se apliquen correctivos de forma selectiva en función de las necesidades de cada zona del campo. Estas requieren técnicas automatizadas, como las de objeto de este trabajo, para la detección y clasificación de malas hierbas.

Algunas técnicas de las propuestas a día de hoy se basan en imágenes RGB que son procesadas y clasificadas por Redes Neuronales Convolucionales de propósito general. Estas, a pesar de su gran rendimiento en este tipo de tareas, presentan grandes deficiencias debido a su arquitectura generalista, como su tamaño, el gran coste computacional o los grandes conjuntos de imágenes etiquetadas que son necesarias para su entrenamiento.

En base a todo lo anterior, este trabajo propone demostrar la hipótesis que, empleando algoritmos de optimización como los Algoritmos Evolutivos, se pueden encontrar arquitecturas más eficientes desde el punto de vista de tamaño, tiempo de entrenamiento y requerimiento de grandes conjuntos de datos, que puedan competir en capacidad de clasificación de imágenes con las arquitectura de las redes que son empleadas en la actualidad.

\newpage

% ------------------------------------------------------------------------------
\vspace*{10mm}

{\usekomafont{chapter}Palabras clave}\label{sec:palabras-clave} \\ 
% ------------------------------------------------------------------------------

Robótica, Visión por computador, Visión artificial, Inteligencia Artificial, Redes Neuronales Artificiales, Redes Neuronales Convolucionales, Algoritmos Evolutivos, Algoritmos Genéticos

% ------------------------------------------------------------------------------
\vspace*{10mm}

{\usekomafont{chapter}Códigos UNESCO}\label{sec:codigos_unesco} \\ 

\begin{itemize}
    \item 120302: Lenguajes Algorítmicos
    \item 120304: Inteligencia Artificial
    \item 120308: Código y Sistemas de Codificación
    \item 120903: Análisis de datos
    \item 120905: Análisis y Diseño de experimentos
    \item 220925: Tratamiento digital de imágenes
    \item 220310: Redes Neuronales
    \item 330419: Robótica
    \item 331101: Tecnología de la automatización
    \item 521201: Agricultura, Silvicultura, Pesca
\end{itemize}

% ------------------------------------------------------------------------------
% %{\usekomafont{chapter}Abstract}\label{sec:abstract2} \\
% % \pdfbookmark[0]{Abstract}{Abstract}

%\cleardoublepage
% \chapter*{Abstract}
% \addcontentsline{toc}{chapter}{ABSTRACT}
% \label{sec:abstract2}
% \vspace*{-10mm}
% % ------------------------------------------------------------------------------

% Aquí va el Resumen en inglés
% [...]


% % ------------------------------------------------------------------------------
% \vspace*{10mm}

% {\usekomafont{chapter}Keywords}\label{sec:keywords} \\ 
% % ------------------------------------------------------------------------------

% Keywords en inglés
% [...]

% % ------------------------------------------------------------------------------