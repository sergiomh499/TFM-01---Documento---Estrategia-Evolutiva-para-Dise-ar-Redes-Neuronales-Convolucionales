% !TEX root = ../my-thesis.tex
%
\chapter{INTRODUCCIÓN}
\label{sec:intro}
    
En este capítulo se tratará de realizar una breve introducción al trabajo realizado. Se presentará la motivación de este trabajo, que objetivos se persiguen y la estructura que mantendrá el documento.\\

\section{Motivación}

La automática y la robótica, son tecnologías cada vez más presentes en gran cantidad de procesos industriales que resultaban ser altamente repetitivos para las personas, consiguiendo una mejora considerable en el rendimiento y la productividad de estos.

Uno de los ámbitos donde está entrando con fuerza en los últimos años es el de la agricultura, donde existen espacios de mejora debido al desarrollo de las nuevas tecnologías y a la aparición de grandes intereses económicos, sociales y medioambientales, propician un gran avance de esta. Una rama de la agricultura que fomenta la introducción de este tipo de desarrollos, es la agricultura de precisión, que promete explotar los cultivos consiguiendo un mayor desempeño de estos.

Dentro de la agricultura de precisión se pueden encontrar tareas como la detección, clasificación y eliminación de malas hierbas, que vienen a mejorar las técnicas extensivas actuales donde se localizan grandes deficiencias que afectan a los costes, a la productividad y a la calidad de la cosecha.

En las fases de detección y clasificación, actualmente se cuentan con varias soluciones, donde las más extendidas se basan en el uso de Redes Neuronales Convolucionales genéricas, debido a su gran desempeño. Esta solución, cuenta con varios desventajas, como son el tamaño de estas redes, grandes requerimientos computacionales y la necesidad de enormes conjuntos de datos de entrenamiento.

Por lo tanto, este trabajo viene a dar respuesta a esta problemática, tratando de diseñar nuevas arquitecturas que se adapten de mejor manera al problema de clasificación propuesto, empleando técnicas de optimización, como son las basadas en estrategias evolutivas. Estas se esperan que suplan las deficiencias encontradas en las arquitecturas de las Redes Neuronales Convolucionales más extendidas en la actualidad.

\section{Objetivos}

Este trabajo tiene como objetivo final la creación de un algoritmo basado en estrategias evolutivas para diseñar redes neuronales convolucionales, además de su ejecución y estudio de los resultados obtenidos para la verificación de la hipótesis realizada.

Para la obtención del objetivo final anteriormente comentado, se establecen una serie de objetivos parciales que han de ser alcanzados:

\begin{itemize}

    \item Familiarización con el desarrollo de técnicas de optimización basadas en estrategias evolutivas.
    
    \item Estudio de la arquitecturas y funcionamiento de las redes neuronales convolucionales desarrolladas en la actualidad.
    
    \item Diseño y optimización de algoritmo de generación de arquitecturas de redes neuronales convolucionales basado en estrategias evolutivas.
    
    \item Implantación y ejecución del algoritmo en infraestructura de computación de alto rendimiento de forma remota.
    
    \item Análisis y comparación de los individuos resultado de la ejecución del algoritmo.
    
    \item Verificación de la hipótesis establecida sobre el método de diseño propuesto.
    
\end{itemize}

\section{Estructura}

La estructura de este documento, se basa en seis puntos diferenciados, donde se pretende dar una visión global del trabajo y de las investigaciones realizadas durante la elaboración del mismo.

\begin{itemize}

    \item En primer lugar, se presenta el \textbf{estado del arte} de las problemáticas actuales existentes en la agricultura. Más específicamente, se desarrolla sobre las técnicas de clasificación de malas hierbas que se localizan bajo el paraguas de la agricultura de precisión, localizando las deficiencias que estas tienen.
    
    \item En segundo lugar, se introducen unas \textbf{bases teóricas} que serán necesarias para la comprensión en mayor o menor medida del trabajo desarrollado en este documento.
    
    \item En tercer lugar, una vez conocidas como funcionan los algoritmos que son estudiados, se localiza \textbf{el problema} que motiva el trabajo y se desarrolla la hipótesis que ha de ser verificada durante el desarrollo de este.
    
    \item En cuarto lugar, se realiza el desarrollo de \textbf{la implementación} realizada para la resolución de este problema, tratando de hacer viable la ejecución posterior de esta para la verificación de la hipótesis propuesta.
    
    \item En quinto lugar, se realizan una serie de \textbf{experimentos} y se comentan los \textbf{resultados} obtenidos de la ejecución de la implementación realizada, de manera que se respondan las cuestiones que motivaban el desarrollo de este trabajo.
    
    \item Finalmente, se presentan \textbf{las conclusiones}, donde se exponen las lecciones aprendidas del desarrollo y obtención de resultados durante la investigación, presentando además unas \textbf{líneas futuras}, donde se expliquen que partes de esta podrían ser interesantes de ampliación o estudio en más profundidad.
    
\end{itemize}
